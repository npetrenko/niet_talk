\subsection{Статистическая модель}
\begin{frame}
\frametitle{Модель цикла}

\includegraphics[scale=0.18]{images/cycle1}
\begin{itemize}
	\item $D_{fm}$ - дамми-переменная $m$-го месяца до выборов в регионе фирмы $f$
	\item $D\_Proc\_Rev_{fe}$ - индикатор "более 1\% оборота фирмы составляют гос заказы"
	\item $Rev_{f}$ - оборот фирмы $f$ в 2003 году (из Росстата)
	\item $Inc\_trans_{f,t}$ - объем входящий трансакций для фирмы $f$ в \\неделю $t$
\end{itemize}
\end{frame}

\begin{frame}
%\includegraphics[scale=0.2]{images/el_effect_1}
\begin{itemize}
	\item $\beta^1_m$:\\
		\includegraphics[scale=0.15]{images/el_effect_beta1}
	
	\item $\beta^2_m$:\\
		\includegraphics[scale=0.15]{images/el_effect_beta2}
\end{itemize}
\end{frame}

\subsection{Механизм действия}

\begin{frame}
\frametitle{Механизм влияния выборов на объем выводимых денег}
\begin{itemize}
\item Коррупция при розыгрыше тендеров:
	\begin{itemize}
	\item Необходимость подкупить действующее должностное лицо
	\item Необходимость повлиять на выборы в пользу лояльных кандидатов
	\end{itemize}

\item Увеличение экономической активности в целом, которая поднимается вследствие желания текущих должностных лиц быть переизбранными

\item Чувствительность склонности к выводу денег к доходам от гос. заказов

\item Риски, связанные с выборами - изменение политики распределения тендеров
\end{itemize}
\end{frame}
