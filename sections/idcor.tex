\subsection{Фирмы-однодневки}

\begin{frame}

\frametitle{Фирмы-однодневки}
В гос тендерах могут участвовать только компании с относительно законной финансовой деятельностью. Поэтому для вывода денег необходимо создавать фирмы-однодневки, якобы оказывающие платные консальтутационные услуги.
\vspace{3mm}

Не имеют никакой формальной связи с менеджментом компании (например, при их регистрации часто используются поддельные документы), и имеют целью своего существования обналичивание денег
\vspace{3mm}
\end{frame}


\begin{frame}
Авторы считают однодневками фирмы, которые удовлетворяют следующим критериям:
\begin{itemize}
\item Объем уплаченных налогов пренебрежимо мал по сравнению с прибылью (менее 0.1\%)
\item Объем уплаченных социальных налогов меньше, чем налоги на одного работника с минимальной з/п
\item Прибыль положительна
\end{itemize}
\vspace{3mm}

Трансакции в пользу фирм-однодневок, проводимые компанией, используется как прокси-переменная для объема взяток должностным лицам. 
\vspace{3mm}

Дальше мы приведем доводы автором в пользу этого решения

\end{frame}