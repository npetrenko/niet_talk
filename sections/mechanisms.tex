
\begin{frame}
\frametitle{Гипотеза влияния предвыбоной активности на цикл вывода денег}
\begin{itemize}
	\item В регрессию включены объемы входящий транзакций за несколько предыдущих периодов, что уже должно значительно снизить влияния этого фактора на $\gamma$
	\item Оценивается регрессия того же вида, но в левой чисти стоят трансферы в пользу законных фирм, не связаных с предвыборной деятельностью
\end{itemize}
\end{frame}

\begin{frame}
$\beta^1_m + \beta^2_m$:
\includegraphics[scale=0.25]{images/legit_trans}
\end{frame}

\begin{frame}
\frametitle{Чувствительность вывода к доходам от гос заказов}
\includegraphics[scale=0.3]{images/tunnelling_from_procurement}
$k$ большие, но это ожидаемо - должностные лица могут требовать взятки после выигрыша тендера
\end{frame}

\begin{frame}
Но в самих доходах от гос закупок политического цикла нет:
\includegraphics[scale=0.3]{images/tunnelling_from_procurement2}\\
\vspace{3mm}

$\delta^1_m$:\\
\includegraphics[scale=0.25]{images/tunnelling_from_procurement3}
\vspace{3mm}

Также добавляют лаггированные $\frac{ProcRev_{ft}}{Rev_f}$ в основное уравнение - результаты оценки не меняются
\end{frame}