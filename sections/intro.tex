\subsection{Цели и результаты}

\begin{frame}

\frametitle{Введение}
Цели исследования:\\ 

\begin{itemize}
\item анализ влияния взяток на выигрыш тендеров на госзакупки

\item проверка гипотезы о том, 
что взятки являются "смазкой" бюрократической системы и помогают делать выбор в пользу
эффективных фирм-исполнителей
\end{itemize}

\vspace{5mm}

Результаты:
\begin{itemize}

\item выявление коррупции в гос закупках

\item построение объективного индекса
коррупции для субрегионального уровня

\item доказательство негативного влияния коррумпированности на
эффективность

\end{itemize}

\end{frame}



\subsection{Данные}

\begin{frame}

\frametitle{Данные}
Утекшие из ЦБ в 2005 данные по банковским трансакциям 1999-2004 годов
\vspace{3mm}

В данный момент находятся в свободном доступе
\vspace{3mm}

Результаты исследований, проведенных с помощью этих данных, 
обсуждаются официальными представителями ЦБ,
что говорит в пользу их релевантности
\vspace{3mm}

Не анализируется участие в розыгрыше тендеров следующих компаний:
\begin{itemize}
\item Гос. компании и компании, полностью принадлежащие государству
\item Компании с месячным оборотом менее 100 тыс. рублей
\item Финансовые компании
\end{itemize}

\end{frame}